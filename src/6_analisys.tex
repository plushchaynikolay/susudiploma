\section{АНАЛИЗ ПРОЕКТОВ}
    \subsection{Метрики}
    % TODO Метрики
    Целью реинжениринга проекта было улучшение его ясности, надежности и поддерживаемости,
    модифицируемости, для этого нужно максимально уменьшить сложность и сцепленность кода,
    увеличить покрытие кода тестами.
    Для оценки успешности выполненных работ, применим метрики размера,
    стилистики и сложности программы.
    \cite{metrics.cmcons}

    Для оценки изменений в объеме кодовой базы подсчитаем количества строк
    (LOC-метрика, \textit{lines of code}) кода в проекте.
    Для оценки стилистики и понятности подсчитаем количество модулей и средний
    объем модуля -- это даст нам понять, насколько изменилась читаемость кода,
    ведь чем больше текст программы, тем сложнее удержать во внимании суть при её
    изучении. Также подсчитаем количество пустых строк, чтобы оценить
    "разряженность" текста программы.
    При этом мы не станем подсчитывать степень документированности кода,
    так как в текущей команде разработчиков не поощряется комментирование.
    Такое решение было принято из тех принципов,
    что код должен быть понятен без комментариев,
    что отсутствие комментариев вынуждает программиста придумывать лучшие имена
    для переменных и функций,
    что комментарии необходимо регулярно актуализировать, а делать это
    гораздо сложнее, чем код, так как комментарии нельзя протестировать.
    Вдобавок, в рекомендациях PEP отмечается, что отсутствующий комментарий лучше неправильного.

    \subsection{Сравнение}
    % TODO Сравнение
    
    \begin{table}[H]
        \caption{Сравнение метрик программного обеспечения проектов}
        \begin{center}
            \begin{tabular}{l|r|r}
                \textbf{Метрика} & \textbf{Исходный} & \textbf{Конечный} \\
                \hline
                Количество строк кода               & 7436 & 5886 \\
                Количество пустых строк             & 1077 & 871 \\
                Разряженность, \%                   & 14.48 & 14.8 \\
                Количество модулей                  & 67 & 87 \\
                Среднее число строк на модуль       & 110.99 & 67.66 \\
                Количество классов и интерфейсов    & 100 & 182 \\
                Цикломатическая сложность           & & \\
                Анализ функциональных точек         & & \\
                Покрытие кода тестированием         & & \\
                Покрытие требований                 & & \\
                Метрики программного пакета от      & & \\
                Роберта Сесиль Мартина              & & \\
                Связность                           & & \\
            \end{tabular}
        \end{center}
    \end{table}

    \subsection*{Вывод по главе 6}
    % TODO Вывод по главе 6
