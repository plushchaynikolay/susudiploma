\addcontentsline{toc}{section}{ВВЕДЕНИЕ}
\section*{ВВЕДЕНИЕ}
Обработка естественных языков (\textit{Natural Language Processing, NLP})
-- направление компьютерной лингвистики и машинного обучения,
изучающее анализ и синтез естественных языков.
Под обработкой естественных языков  понимается создание систем,
обрабатывающих естественные языки с целью выполнения определенных задач.
В список этих задач могут входить \cite{cursera.nlp}:
\begin{itemize}
    \item Формирование ответов на вопросы (\textit{Question Answering})
    \item Анализ эмоциональной окраски высказываний (\textit{Sentiment Analysis})
    \item Машинный перевод (\textit{Machine Translation})
    \item Распознавание речи (\textit{Speech Recognition})
    \item Морфологическая разметка (\textit{Part of Speech Tagging})
    \item Извлечение сущностей (\textit{Name Entity Recognition})
\end{itemize}

Диалоговые системы -- это системы, решающие задачу формирования ответов на вопросы.
Само формирование ответов может сопровождаться решением сопутствующих задач, таких как
извлечение именованных сущностей или анализ эмоциональной окраски.
Диалоговые системы можно охарактеризовать по двум векторам:
задача-ориентированная или общего назначения (\textit{task-oriented/general})
и закрытого или открытого диапазона тем (\textit{closed/open domain}).
Первая классификация определяет создана ли система для решения одного конкретного
вопроса по данной теме в ходе диалога, или позволяет свободно вести продолжительный
диалог на поставленную тему.
Вторая классификация определяет на сколько диалоговая система ограничена в диапазоне выбора тем
для беседы с человеком.

Чат-боты -- это диалоговые системы, обладающие интерфейсом для взаимодействия с человеком.
Чаще всего чат-боты работают через мессенджеры, социальные сети, форумы или диалоговые окна на сайтах.
Также, чаще всего боты используются для решения типовых задач, хорошо описываемых сценариями.
Такие чат-боты часто применяются в бизнесе для оперативного решения вопросов клиентов.
Чат-боты со свободной темой диалога чаще используются для развлечения.

Клиентские чат-боты предназначены для взаимодействия с клинтами той или иной организации,
оперативного и массового решения вопросов клиента, либо сбора информации
для оформления лучшего предложения.

Чат-бот, рассматриваемый в этой работе, относится к клиентским чат-ботам,
имеет четкий список тем для диалогов и шаблоны ответов, что позволяет отнести
его к группе, так называемых, авто-информаторов.

В рамках данной работы, проводится анализ результатов рефакторинг сервиса чат-бота.
% Поскольку
% для обновленного сервиса будет использоваться иная архитектура, другие библиотеки и
% фреймворки, правильнее будет назвать проводимые работы реинжинирингом.

Рефакторинг -- это изменение внутренней структуры программного обеспечения,
имеющее целью облегчить понимание его работы и упростить модификацию,
не затрагивая наблюдаемого поведения. \cite{refactoring.fowler}

Во время жизненного цикла программного обеспечения, его код претерпевает изменения
с той или иной частоте, в тех или иных масштабах. В это время код теряет свою
структурированность, обрастает сложными конструкциями, повторяющимися кусками кода,
незадокументированными фрагментами.
Рефакторинг позволяет улучшить композицию и структуру и кода, уменьшить дублирование,
облегчить понимание, провести профилактику ошибок, улучшить модифицируемость и
уменьшить время внесения новых изменений.


% Иногда может наступить критический момент, когда команда понимает,
% что при проектировании системы были совершены существенные ошибки,
% что текущее состояние кода не может быть исправлено очередным рефакторингом,
% но может создать препятствия к дальнейшему развитию проекта,
% или уже создает затруднения в поддержании рабочего состояния.
% В этом случае уместным может быть перепроектировать систему на других основах,
% то есть произвести реинжиниринг.
