\addcontentsline{toc}{section}{ВВЕДЕНИЕ}
\section*{ВВЕДЕНИЕ}
Обработка естественных языков (\textit{Natural Language Processing, NLP})
-- направление компьютерной лигвистики и машинного обучния,
изучающее анализ и синтез естественных языков.
Под обработкой естественных языков  понимается создание систем,
обрабатывающих естественные языки с целью выполнения определенных задач.
В список этих задач могут входить:
\begin{itemize}
    \item Формирование ответов на вопросы (\textit{Question Answering})
    \item Анализ эмоциональной окраски высказываний (\textit{Sentiment Analysis})
    \item Машинный перевод (\textit{Machine Translation})
    \item Распознавание речи (\textit{Speech Recognition})
    \item Морфологическая разметка (\textit{Part of Speech Tagging})
    \item Извлечение сущностей (\textit{Name Entity Recognition})
\end{itemize}

Диалоговые системы -- это системы, решающие задачу формирвания отвеов на вопросы.
Само формирование ответов может споровождаться решением сопутствующих задач, таких как
извлечение именнованных сущностей или анализ эмоциональной окраски.
Диалоговые системы можно охарактеризовать по двум векторам:
задача-ориентированная или общего назначения (\textit{task-oriented/general})
и закрытого или открытого диапазона тем (\textit{closed/open domain}).
Первая классификация определяет создани ли система для решения одного конкретного
вопроса по данной теме в ходе диалога, или позваляет свободно вести продолжительный
диалог на поставленную тему.
Вторая классификация определяет на сколько диалоговая система ограничена в диапазоне выбора тем
для беседы с человеком.

Чат-боты -- это диалоговые системы, обладающие интерфесом для взаимодействия с человеком.
Чаще всего чат-боты работают через мессенджеры, социальные сети, форумы или диалоговые окна на сайтах.
Также, чаще всего боты используются для решения типовых задач, хорошо описываемых сценариями.
Такие чат-боты часто применяются в бизнесе для оперативного решения вопросов клиентов.
Чат-боты со свободной темой диалога чаще используются для развлечения.

Клиентские чат-боты предназначены для взаимодействия с клинтами той или иной организации,
оперативного и массового решения вопросов клиента, либо сбора информации
для оформления лучшего предолжения.

Чат-бот, рассматриваемый в этой работе, относится к клиентским чат-ботам,
имеет четкий список тем для диалогов и шаблоны ответов, что позволяет отнести
его к группе, так называемых, авто-информаторов.
