\section{ТЕСТИРОВАНИЕ}
    \subsection{Статический анализ}
    % TODO Статический анализ
    Для статического анализа соответствия типов в Python используются аннотации 
    из встроенного модуля \inlinecode{typing} и анализатор.
    Существует несколько анализаторов, \textit{mypy} - наиболее популярный из них,
    написан на Python, поддерживается и развивается сообществом
    и Гвидо ван Россумом в частности.

    \subsection{Юнит-тесты}
    % TODO Юнит-тесты

    \subsection{Нагрузочное тестирование}
    Нагрузочное тестирование проводилось на тестовом сервере на полностью
    развернутом веб-приложении с веб-сервером \textit{Nginx} при помощи
    инструмента Яндекс.Танк. Целью нагрузочного тестирования было установление
    порога стабильной работы сервиса, для этого на него подавалась нагрузка
    с константным количеством запросов в секунду в течение 2 минут с итеративным
    увеличением количества запросов.

    Так, был составлен следующий конфигурационный файл для Яндекс.Танка:
    \begin{figure}[!h]
        \centering
        \lstinputlisting[language=Python]{snippets/load.yaml}
        \caption{Конфигурационный файл Яндекс.Танка}
        \label{fig:tank_load}
    \end{figure}

    Так, было установлено, что сервис стабильно выдерживает нагрузку до 18
    запросов в секунду в течение 2 минут.
    % TODO какой-нибудь график
