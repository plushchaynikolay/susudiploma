\section{РЕАЛИЗАЦИЯ}
    \subsection{Уведомления}
    Выполнение несрочных задач в проекте выполняется менеджером отложенных задач \textit(Celery):
    это широко известный кроссплатформенный проект с открытым исходным кодом на Python для
    асинхронного выполения задач через вызов удаленных процедур. В качестве брокера задач мы используем RabbitMQ.
    Для добавления задач в очередь в нашем проекте используются "уведомления" (\textit{Notifier}).
    Перечислим все виды уведомлений, которые существут на настоящий момент:
    \begin{itemize}
        \item \textit{DataBaseMessageNotifier} - сохранение сообщения в базу;
        \item \textit{DataBaseResponseNotifier} - сохранение ответов в базу;
        \item \textit{DataBaseCarefulNotifier} - сохранение сообщения в базу, если оно ещё не существует;
        \item \textit{AppLogerNotifier} - уведомление о некоторых событиях в мессенджер.
    \end{itemize}
    Пример реализации сохрания в базу представлен на листинге ниже.
\begin{lstlisting}[language=Python]
class DataBaseMessageNotifier(Notifier):
    @staticmethod
    def notify(message: MessageData,
               scenarist_response: Optional[ResponseData] = None) -> None:
        celery_app.send_task('chat_bot.tasks.save_to_db',
                             args=(message.to_dict(),))
\end{lstlisting}

    Сама Celery-задача выглядит следующим образом:
\begin{lstlisting}[language=Python]
@celery_app.task(name='chat_bot.tasks.save_to_db')
def save_to_db(json_message: Mapping[str, Any]) -> None:
    message = MessageData.from_dict(json_message, [EmailMessageData])
    DBProvider.save(message)
\end{lstlisting}

    \subsection{Работа с базой}
    В работе с базой данных участвуют две сущности: объектно-реляционная модель Message
    и класс инкапсулирующий работу с базой DBProvider.
    Модель Message выглядит следующим образом:
\begin{lstlisting}[language=Python]
class Message:
    class Meta:
        db_table = 'messages'
        constraints = [
            models.UniqueConstraint(
                fields=['source', 'source_message_id'],
                name='messages_source_source_message_id_key')
        ]
    id = models.AutoField(primary_key=True)
    chat_id = models.CharField(max_length=50, null=True)
    source = models.CharField(max_length=50, null=False)
    datetime = models.DateTimeField(null=False)
    text = models.TextField(null=True)
    sender = models.CharField(max_length=10, null=False)
    is_answered = models.BooleanField(null=True)
    prediction = JSONField(null=True)
    correctly_answered = models.BooleanField(null=True)
    categories = ArrayField(models.IntegerField(), null=True)
    source_message_id = models.CharField(max_length=50, null=True)
\end{lstlisting}

    DBProvider - это довольно большой статический класс, реализующий как получение исории сообщений,
    так и сохранение новых сообщений в базу.

    \subsection{Датаклассы}
    \subsection{Канал Телеграма}
    \subsection{Канал Вконтакте}
    \subsection{Канал почты}
    \subsection{Сцерарист и сценарии}
    \subsection{Обработчик}
    \subsection*{Вывод по главе 3}
