\section{РЕАЛИЗАЦИЯ}
    \subsection{Уведомления}
    Выполнение несрочных задач в проекте выполняется менеджером отложенных задач \textit(Celery):
    это широко известный кроссплатформенный проект с открытым исходным кодом на Python для
    асинхронного выполения задач через вызов удаленных процедур. В качестве брокера задач мы используем RabbitMQ.
    Для добавления задач в очередь в нашем проекте используются "уведомления" (\inlinecode{Notifier}).
    Перечислим все виды уведомлений, которые существуют на настоящий момент:
    \begin{itemize}
        \item \inlinecode{DataBaseMessageNotifier} -- сохранение сообщения в базу;
        \item \inlinecode{DataBaseResponseNotifier} -- сохранение ответов в базу;
        \item \inlinecode{DataBaseCarefulNotifier} -- сохранение сообщения в базу, если оно ещё не существует;
        \item \inlinecode{AppLogerNotifier} -- уведомление о некоторых событиях в мессенджер.
    \end{itemize}
    Пример реализации сохрания в базу представлен на листинге ниже.
\begin{lstlisting}[language=Python]
class DataBaseMessageNotifier(Notifier):
    @staticmethod
    def notify(message: MessageData,
               scenarist_response: Optional[ResponseData] = None
               ) -> None:
        celery_app.send_task('chat_bot.tasks.save_to_db',
                             args=(message.to_dict(),))
\end{lstlisting}

    Сама Celery-задача выглядит следующим образом:
\begin{lstlisting}[language=Python]
@celery_app.task(name='chat_bot.tasks.save_to_db')
def save_to_db(json_message: Mapping[str, Any]) -> None:
    message = MessageData.from_dict(json_message, [EmailMessageData])
    DBProvider.save(message)
\end{lstlisting}

    \subsection{Работа с базой}
    В работе с базой данных участвуют две сущности: объектно-реляционная модель Message
    и класс инкапсулирующий работу с базой DBProvider.

    \newpage
    Модель Message выглядит следующим образом:
\begin{lstlisting}[language=Python]
class Message:
    class Meta:
        db_table = 'messages'
        constraints = [
            models.UniqueConstraint(
                fields=['source', 'source_message_id'],
                name='messages_source_source_message_id_key')
        ]
    id = models.AutoField(primary_key=True)
    chat_id = models.CharField(max_length=50, null=True)
    source = models.CharField(max_length=50, null=False)
    datetime = models.DateTimeField(null=False)
    text = models.TextField(null=True)
    sender = models.CharField(max_length=10, null=False)
    is_answered = models.BooleanField(null=True)
    prediction = JSONField(null=True)
    correctly_answered = models.BooleanField(null=True)
    categories = ArrayField(models.IntegerField(), null=True)
    source_message_id = models.CharField(max_length=50, null=True)
\end{lstlisting}

    DBProvider -- это довольно большой статический класс, реализующий как получение исории сообщений,
    так и сохранение новых сообщений в базу.
    Структура класса приведена на листинге ниже.
\begin{lstlisting}[language=Python]
class DBProvider:
    def get_history(cls, chat_id: str, 
                    columns: Iterable[str]) -> Optional[History]: ...
    def _load_history(chat_id: str, 
                      columns: Iterable[str]) -> History: ...
    def get_latest_datetime(**kwargs: str
                            ) -> Optional[timezone.datetime]: ...
    def save(cls, message: MessageData) -> None: ...
    def get_or_create(cls, message: MessageData) -> bool: ...
    def update_or_create(cls, message: MessageData) -> bool: ...
    def _save(message: MessageData, 
              need_check: bool = False, 
              update: bool = False) -> bool: ...
\end{lstlisting}

    Рассмотрим подробнее методы этого класса:
    \begin{itemize}
        \item \inlinecode{get_history} -- публичный метод для получения истории диалога, осуществляет
        безопасную выгрузку истории чата из базы данных, обрабатывая возникающие исключения;
        \item \inlinecode{_load_history} -- приватный метод, реализующий непосредственную загрузку истории из базы,
        устанавливает временной диапазон истории и формирует результат в надлежащий вид;
        \item \inlinecode{get_latest_datetime} -- метод получения информации о времени последнего сообщения
        в канале или чате, от пользователя, оператора или бота;
        \item \inlinecode{save} -- сохранение нового сообщения в базе, реализует SQL-функцию \inlinecode{INSERT};
        \item \inlinecode{get_or_create} -- создание записи о сообщении в базе, если в этом канале сообщения с таким
        идентификатором ещё не существует, реализует SQL-функцию \inlinecode{IF NOT EXISTS INSERT};
        \item \inlinecode{update_or_create} -- обновление записи в базе о сообщении определенного канала с таким
        идентификатором, либо создание новой записи, реализует SQL-функцию \inlinecode{IF EXISTS UPDATE ELSE INSERT};
        \item \inlinecode{_save} -- непосредственная реализация трех перечисленных выше функций,
        выбирая нужный метод библиотеки \textit{Django} подставляет нужные фильтры поиска и значения для создания
        или обновления записи.
    \end{itemize}

    \subsection{Датаклассы}
    С версией \textit{Python 3.8} в синтаксис языка были введены так называемые
    датаклассы (\textit{dataclasses}).
    Это усовершенствованный синтаксис для описания собственных типов и структур данных.
    В проекте существуют несколько представителей датаклассов. Опишем отдельно каждого из них.
    \begin{enumerate}
        \item \inlinecode{MessageData} -- датакласс для описания сообщений входящих и исходящих,
        содержит всю необходимую для работы обработчика, сценариста и других классов информацию:
        идентификатор чата, название канала-источника, время получения, отправителя,
        список классифицированных категорий, а также некоторые флаги и идентификаторы.
\begin{lstlisting}[language=Python]
@dataclass
class MessageData(SerializableMixin):
    chat_id: str = ''
    source: str = ''
    timestamp: Optional[datetime] = None
    text: str = ''
    sender: Optional[str] = None
    is_answered: Optional[bool] = None
    categories: List[int] = field(default_factory=list)
    prediction: Optional[Dict[str, float]] = None
    source_message_id: Optional[str] = None
    source_chat_id: Optional[str] = None
\end{lstlisting}

        \item \inlinecode{Answer} -- контейнер с ответом бота на входящее сообщение.
\begin{lstlisting}[language=Python]
@dataclass
class Answer(SerializableMixin):
    category: str
    message: MessageData
\end{lstlisting}

        \item \inlinecode{Prediction} -- контейнер с результатами классификации текста сообщения,
        которые возвращает классификатор.
\begin{lstlisting}[language=Python]
@dataclass
class Prediction(SerializableMixin):
    categories: List[str]
    probabilities: Optional[Dict[str, float]]
\end{lstlisting}

        \item \inlinecode{ResponseData} -- датакласс со всем, что нужно боту, чтобы корректно
        среагировать на входящее сообщение пользователя.
\begin{lstlisting}[language=Python]
@dataclass
class ResponseData(SerializableMixin):
    prediction: Prediction = field(default=Prediction([], None))
    answers: List[Answer] = field(default_factory=list)
    after_answer_actions: List[Action] = field(
        default_factory=list)
\end{lstlisting}
    \end{enumerate}

    \subsection{Канал Телеграма}
    Так как за обработку отложенных действий теперь отвечает менеджер задач \textit{Celery},
    исчезла потребность в асинхронной обработке сообщений, мы отказались от асинхронности.
    Это позволяет использовать синхронные библиотеки, выбор которых гораздо больше,
    а логика синхронных приложений гораздо проще для понимания.
    В новой реализации Телеграм-канала была использована другая библиотека -- \textit{pyrogram}.
    Структура \inlinecode{TelegramWorker} осталась примерно той же: два фильтра событий и две функции, описывающие
    обработку входящих и исходящих сообщений подаются в цикл обработки событий (\textit{event loop})
    телеграм клиента.
    Обработка сообщений заключается в их преобразовании во внутренний тип \inlinecode{MessageData}
    и передаче обработчику в \inlinecode{IncomingHandler} или в \inlinecode{OutgoingHandler}.
    \inlinecode{TelegramApiWrapper} наследуется от базового класса \inlinecode{ApiWrapper},
    реализует его абстрактные методы \inlinecode{send_message} и \inlinecode{convert_from_api_message_type}.

    \subsection{Канал Вконтакте}
    Для реализации канала Вконтакте используется та же библиотека \textit{vk\_api}. Для получения сообщений используется
    технология LongPoll, в которой, после установления соединения и начала сессии, клиент получает
    от сервера Вконтакте уведомления о всех новых событиях в беседе. У каждого нового события проверяется
    является ли оно сообщением, а также поле \inlinecode{to_me}, из которого определяется каким образом
    должно обрабатываться сообщение, как входящее или исходящее.
    Обработка сообщений, также как и в случае с телеграмом, заключается в их преобразовании во внутренний тип
    и передаче одному из обработчиков.

    \subsection{Канал почты}
    Ввиду отсутствия у библиотек по работе с почтой инструментов, похожих на LongPoll, этот инструмент был
    создан самостоятельно. Для этого мы поочередно просматриваем папки входящей и исходящей почты на наличие
    непрочитанных и необработанных ботом сообщений. Также был объявлен специальный датакласс \inlinecode{EmailMessage}
    канала почты, со специальными полями, и класс \inlinecode{EmailMessageBuilder} для создания объекта датакласса.
    Дополнительно были созданы классы обертки над библиотеками \textit{imap} и \textit{smtp}, в которых
    реализован алгоритм установления и проверки соединения и переподключения в случае его разрыва.
    На почте присутсвует сложная логика, связанная пометкой обработанных ботом сообщений флажком
    и перекладыванием отвеченных сообщений в отдельную папку. Дополнительно ситуация осложняется тем, что
    одновременно в одном почтовом ящике могут работать кроме бота ещё два оператора и сторонний парсер почты,
    из-за чего часто возникают конфликтные ситуации. Ситуации могут выражаться, например в прочтении и снятии флага
    \inlinecode{unseen} с писем, предназначенных для бота, и так далее. Поэтому обработка писем в этом канале перегружена
    различными дополнительными проверками.

    \subsection{Сценарист и сценарии}
    Диалоговая система в целом не перенесла существенных изменений, так как целью рефакторинга не ставилось
    улучшение точности или полноты распознавания. Изменения претерпел только модуль для сборки модели.
    Он был перенесен в рабочий проект целиком, практически без изменений, в корневую директорию нового проекта.
    В перую очередь, это было нужно, чтобы решить проблему импортов библиотек при распаковке модели.
    Дело в том, что для сериализации модели в проекте используется библиотека \textit{joblib}, которая при сериализации
    классов, использованных в составе модели, сохраняет путь импортов, а при десериализации использует эти же пути
    для восстановления модели в оперативной памяти.
    Проблема заключается в том, что пути импортов сохраняются отностильно точки запуска скрипта сборки, или,
    в нашем случае, \textit{jupyter} тетрадки с кодом сборки. Соответственно, если точка входа в приложение
    будет находится в другом каталоге, импорты провалятся и приложение завершится с ошибкой.
    Размещение скриптов для сборки модели и запуска приложения в одном каталоге -- самое легкое и очевидное
    решение, в дальнейшем позволяет не отвлекаться на проблему загрузки модели во время работы над ней.

    \subsection{Обработчик}
    Для обработки сообщений был реализован базовый класс \inlinecode{MessageHandler}, в котром объявлен
    абстрактный метод \inlinecode{handle()} и реализован общий для входящих и исходящих сообщений
    метод \inlinecode{notify}, который рассылает уведомления всем заявленным уведомителям.
    Метод \inlinecode{handle()} содержит реализуется в виде перечня последовательных действий,
    которые должны быть проделаны над сообщением.

    В \inlinecode{IncomingHandler} это вызов сценариста для формирования ответа, отправка ответных сообщений,
    отправка уведомлений, выполнение дополнительных действий. Листинг метода представлен ниже.
\begin{lstlisting}
class IncomingHandler(MessageHandler):
    def handle(self, message: IMessageData,
               api_wrapper: ApiWrapper[IMessageData],
               notifiers: Sequence[Type[Notifier]] = ()) -> None:
        response = self.predict(message)
        response.answers = self.send_answers(response.answers, api_wrapper)
        self.notify(notifiers, message_with_prediction, response)
        self.do_actions(response.after_answer_actions, api_wrapper)
\end{lstlisting}

    В \inlinecode{OutgoingHandler} никаких действий кроме отправки уведомлений не требуется.
    Реализация этого класса представлена ниже.
\begin{lstlisting}
class OutgoingHandler(MessageHandler):
    def handle(self, message: IMessageData,
               notifiers: Sequence[Type[Notifier]] = ()) -> None:
        self.notify(notifiers, message)
\end{lstlisting}

    \subsection*{Вывод по главе 3}
    В этой главе мы описали все реализованные нами классы и методы, объяснили принципы
    их работы и перечислили некоторые связанные с ними нюансы. Для наглядности мы привели
    куски кода.
    В ходе самой разработки были решены основные проблемы, перечисленные в первой главе данной работы.
    Проект переписан с единой архитектурой для всех каналов чат-бота что улучшает понимание и поддержку
    проекта, для работы с базой теперь используется объектно-реляционная модель, что также сильно
    упрощает поддержку и написание нового функционала,
    проект полностью снабжен аннотациями типов, что позволяет проводить статический анализ.
