\section{РЕАЛИЗАЦИЯ}
    \subsection{Уведомления}
    Выполнение несрочных задач в проекте выполняется менеджером отложенных задач \textit(Celery):
    это широко известный кроссплатформенный проект с открытым исходным кодом на Python для
    асинхронного выполения задач через вызов удаленных процедур. В качестве брокера задач мы используем RabbitMQ.
    Для добавления задач в очередь в нашем проекте используются "уведомления" (\inlinecode{Notifier}).
    Перечислим все виды уведомлений, которые существуют на настоящий момент:
    \begin{itemize}
        \item \inlinecode{DataBaseMessageNotifier} - сохранение сообщения в базу;
        \item \inlinecode{DataBaseResponseNotifier} - сохранение ответов в базу;
        \item \inlinecode{DataBaseCarefulNotifier} - сохранение сообщения в базу, если оно ещё не существует;
        \item \inlinecode{AppLogerNotifier} - уведомление о некоторых событиях в мессенджер.
    \end{itemize}
    Пример реализации сохрания в базу представлен на листинге ниже.
\begin{lstlisting}[language=Python]
class DataBaseMessageNotifier(Notifier):
    @staticmethod
    def notify(message: MessageData,
               scenarist_response: Optional[ResponseData] = None
               ) -> None:
        celery_app.send_task('chat_bot.tasks.save_to_db',
                             args=(message.to_dict(),))
\end{lstlisting}

    Сама Celery-задача выглядит следующим образом:
\begin{lstlisting}[language=Python]
@celery_app.task(name='chat_bot.tasks.save_to_db')
def save_to_db(json_message: Mapping[str, Any]) -> None:
    message = MessageData.from_dict(json_message, [EmailMessageData])
    DBProvider.save(message)
\end{lstlisting}

    \subsection{Работа с базой}
    В работе с базой данных участвуют две сущности: объектно-реляционная модель Message
    и класс инкапсулирующий работу с базой DBProvider.
    Модель Message выглядит следующим образом:
\begin{lstlisting}[language=Python]
class Message:
    class Meta:
        db_table = 'messages'
        constraints = [
            models.UniqueConstraint(
                fields=['source', 'source_message_id'],
                name='messages_source_source_message_id_key')
        ]
    id = models.AutoField(primary_key=True)
    chat_id = models.CharField(max_length=50, null=True)
    source = models.CharField(max_length=50, null=False)
    datetime = models.DateTimeField(null=False)
    text = models.TextField(null=True)
    sender = models.CharField(max_length=10, null=False)
    is_answered = models.BooleanField(null=True)
    prediction = JSONField(null=True)
    correctly_answered = models.BooleanField(null=True)
    categories = ArrayField(models.IntegerField(), null=True)
    source_message_id = models.CharField(max_length=50, null=True)
\end{lstlisting}

    DBProvider - это довольно большой статический класс, реализующий как получение исории сообщений,
    так и сохранение новых сообщений в базу.
    Структура класса приведена на листинге ниже.
\begin{lstlisting}[language=Python]
class DBProvider:
    def get_history(cls, chat_id: str, 
                    columns: Iterable[str]) -> Optional[History]: ...
    def _load_history(chat_id: str, 
                      columns: Iterable[str]) -> History: ...
    def get_latest_datetime(**kwargs: str
                            ) -> Optional[timezone.datetime]: ...
    def save(cls, message: MessageData) -> None: ...
    def get_or_create(cls, message: MessageData) -> bool: ...
    def update_or_create(cls, message: MessageData) -> bool: ...
    def _save(message: MessageData, 
              need_check: bool = False, 
              update: bool = False) -> bool: ...
\end{lstlisting}

    Рассмотрим подробнее методы этого класса:
    \begin{itemize}
        \item \inlinecode{get_history} - публичный метод для получения истории диалога, осуществляет
        безопасную выгрузку истории чата из базы данных, обрабатывая возникающие исключения;
        \item \inlinecode{_load_history} - приватный метод, реализующий непосредственную загрузку истории из базы,
        устанавливает временной диапазон истории и формирует результат в надлежащий вид;
        \item \inlinecode{get_latest_datetime} - метод получения информации о времени последнего сообщения
        в канале или чате, от пользователя, оператора или бота;
        \item \inlinecode{save} - сохранение нового сообщения в базе, реализует SQL-функцию \inlinecode{INSERT};
        \item \inlinecode{get_or_create} - создание записи о сообщении в базе, если в этом канале сообщения с таким
        идентификатором ещё не существует, реализует SQL-функцию \inlinecode{IF NOT EXISTS INSERT};
        \item \inlinecode{update_or_create} - обновление записи в базе о сообщении определенного канала с таким
        идентификатором, либо создание новой записи, реализует SQL-функцию \inlinecode{IF EXISTS UPDATE ELSE INSERT};
        \item \inlinecode{_save} - непосредственная реализация три перечисленные выше функции, выбирая нужный
        метод библиотеки \textit{Django} подставляет нужные фильтры поиска и значения для создания
        или обновления записи.
    \end{itemize}

    \subsection{Датаклассы}
    С версией \textit{Python 3.8} в синтаксис языка были введены датаклассы (\textit{dataclasses}).
    Это усовершенствованный синтаксис для описания собственных типов и структур данных.
    В проекте существуют несколько представителей датаклассов. Опишем отдельно каждого из них.
    \begin{enumerate}
        \item \inlinecode{MessageData} - датакласс для описания сообщений входящих и исходящих,
        содержит всю необходимую для работы обработчика, сценариста и других классов информацию:
        идентификатор чата, название канала-источника, время получения, отправителя,
        список классифицированных категорий, а также некоторые флаги и идентификаторы.
\begin{lstlisting}[language=Python]
@dataclass
class MessageData(SerializableMixin):
    chat_id: str = ''
    source: str = ''
    timestamp: Optional[datetime] = None
    text: str = ''
    sender: Optional[str] = None
    is_answered: Optional[bool] = None
    categories: List[int] = field(default_factory=list)
    prediction: Optional[Dict[str, float]] = None
    source_message_id: Optional[str] = None
    source_chat_id: Optional[str] = None
\end{lstlisting}

        \item \inlinecode{Answer} - контейнер с ответом бота на входящее сообщение.
\begin{lstlisting}[language=Python]
@dataclass
class Answer(SerializableMixin):
    category: str
    message: MessageData
\end{lstlisting}

        \item \inlinecode{Prediction} - контейнер с результатами классификации текста сообщения,
        которые возвращает классификатор.
\begin{lstlisting}[language=Python]
@dataclass
class Prediction(SerializableMixin):
    categories: List[str]
    probabilities: Optional[Dict[str, float]]
\end{lstlisting}

        \item \inlinecode{ResponseData} - датакласс со всем, что нужно боту, чтобы корректно
        среагировать на входящее сообщение пользователя.
\begin{lstlisting}[language=Python]
@dataclass
class ResponseData(SerializableMixin):
    prediction: Prediction = field(default=Prediction([], None))
    answers: List[Answer] = field(default_factory=list)
    after_answer_actions: List[Action] = field(
        default_factory=list)
\end{lstlisting}
    \end{enumerate}

    \subsection{Канал Телеграма}
    \subsection{Канал Вконтакте}
    \subsection{Канал почты}
    \subsection{Сцерарист и сценарии}
    \subsection{Обработчик}
    \subsection*{Вывод по главе 3}
