\begin{titlepage}
    
    \begin{center}
        МИНИСТЕРСТВО И НАУКИ И ВЫСШЕГО ОБРАЗОВАНИЯ РОССИЙСКОЙ ФЕДЕРАЦИИ\\
        Федеральное государственное автономное образовательное\\
        учреждение высшего образования

        \textbf{
        <<Южно-Уральский Государственный университет\\
        (национальный исследовательский университет)>>\\
        Высшая школа электроники и компьютерных наук\\
        Кафедра системного программирования
        }
        \bigskip
    \end{center}

    \hfill
    \begin{minipage}{0.4\textwidth}
        УТВЕРЖДАЮ\\
        Заведующий кафедрой,\\
        \underline{\hspace{2cm}} Л.Б. Соколинский\\
        02.03.2020
    \end{minipage}

    \begin{center}
        \textbf{ЗАДАНИЕ}\\
        \textbf{на выполнение выпускной квалификационной работы магистра}\\
        студенту группы КЭ-220 Плющай Николаю Владимировичу,\\
        02.04.02 «Фундаментальная информатика и информационные технологии»\\
        (магистерская программа «Технологии разработки высоконагруженных систем»)\\
    \end{center}

    \begin{enumerate}
        \item \textbf{Тема работы:}
        Реинжиниринг консультационного чат-бота для клиентов торговой компании.
        \item \textbf{Срок сдачи студентом законченной работы:} 31.05.2020.
        \item \textbf{Исходные данные к работе:}
        \begin{enumerate}
            \item М. Фаулер Улучшение существующего кода // Символ-Плюс, 2003.
            \item С. Макконнел Совершенный код // Microsoft Press, 2017.
            \item Э. Гамма и др. Приемы объектно-ориентированного проектирования // Питер, 2015.
        \end{enumerate}
        \item \textbf{Перечень подлежащих разработке вопросов}
        \begin{enumerate}
            % \item Произвести анализ исходного проекта.
            % \item Выявить проблемные места.
            % \item Составить план проведения работ.
            % \item Изучить современные подходы и средства разработки.
            % \item Спроектировать систему чат-бота.
            \item Протестировать новую систему чат-бота.
            \item Провести анализ результатов рефакторинга.
        \end{enumerate}
        \item \textbf{Дата выдачи задания:} 24.04.2020.
    \end{enumerate}

    \noindent\\
    \textbf{Научный руководитель}\\
    Доцент кафедры СП, к.т.н. \hfill Н.Ю. Долганина
    
    \noindent\\
    \textbf{Задание принял к исполнению} \hfill Н.В. Плющай

    \thispagestyle{empty}
\end{titlepage}