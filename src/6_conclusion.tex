\addcontentsline{toc}{section}{ЗАКЛЮЧЕНИЕ}
\section*{ЗАКЛЮЧЕНИЕ}
    В этой работе мы рассмотрели виды тестирования, реализованные в новом проекте.
    В проекте были использованы модульные, интеграционные, системные и нагрузочные
    тесты для исследования и контроля за соответствием продуктом требуемых
    поведения и производительности,
    что в дальнейшем поможет в развитии и поддержке проекта.
    
    Мы провели анализ сложности и поддерживаемости проектов
    до и после проведения работ по рефакторингу. Из результатов анализа
    можно сделать вывод, что качество кодовой базы очевидно повысилось.
    Хотя общее количество строк кода увеличилось, количество пустых и
    нефункциональных строк стало больше, благодаря разбиению больших функций
    на несколько маленьких, улучшилась их читаемость, их стало проще тестировать
    и изменять.

    Также увеличилось количество модулей, а сами модули стали меньше.
    Как показывает практика, модули крупнее одного разворота экрана, усложняют
    восприятие кода, заставляют программиста держать в голове слишком много
    информации одновременно.
    То же самое касается более компактных классов, ответственность между которыми
    стала лучше распределена, чем в исходном проекте.
    Также снизилась цикломатическая сложность кода проекта, а проддерживаемость
    улучшилась.

    Код проекта теперь покрыт тестами на 79\%, это повышает надежность продукта,
    позволяет раньше обнаружить, если очередные внесенные правки
    изменили поведение или нарушили работоспособность продукта.
    Так как тесты автоматизированы, экономится время программиста на обнаружение
    проблем, которые раньше он мог обнаружить только при отладке вручную.

    % С развитием социальных сетей, их значимость для бизнеса неуклонно растет,
    % равно как растет темп ритма жизни. Бизнес сейчас зачастую теряет прибыль
    % только из-за того, что не может дать клиентам быструю обратную связь.
    % Развитие и продвижение чат-ботов помогает уменьшить время отклика на запрос
    % клиента и решить возникающую проблему.

    % Целью данной работы было проведение работ по улучшению кодовой базы,
    % увеличению надежности, читаемости и модифицируемости существующего проекта чат-бота.
    % Для достижения этой цели был проведен анализ исходного проекта, выявлен ряд
    % его недостатков, составлен план работ с перечнем требуемых характеристик конечного
    % результата, разработана и реализована новая архитектура системы

    % По итогу выполнения работ была получен единый цельный проект, с четко
    % организованной архитектурой,
    % использующий удобные и понятные средства по работе с базой данных
    % и системой отложенного выполнения задач.
    % Поскольку работы велись по принципу разработки через тестирование,
    % проект максимально покрыт
    % модульными, интеграционными и системными тестами.

    % В планах дальнейшего развития проекта находится проведение работ по улучшению
    % архитектуры диалоговой системы чат-бота, которая не была затронута в ходе
    % текущего рефакторинга.
