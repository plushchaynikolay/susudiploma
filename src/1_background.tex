\section{АНАЛИЗ ПРЕДМЕТНОЙ ОБЛАСТИ}
    \subsection{Анализ исходных данных}
    Существующий проект выполняет задачи сбора сообщений с социальных сетей, мессенджеров и почты, их классификации и рассылки ответов пользователям.
    Проект оснащен системой непрерывной сборки и развертывания (CI/CD Pipeline), интегрированной в систему управления репозиториями GitLab.
    Кодовая база проекта, как и сам проект, разделены на пять репозиториев с различными функциональными частями.
    
    Ниже представлен их список:
    \item Приложения по работе с каналами
    \item API чат-бота для внешнего использования
    \item Диалоговая система чат-бота
    \item Система мониторинга и разметки сообщений
    \item Прокси-сервер и мониторинг нагрузки

    Приложения по работе с каналами выполнены на языке Python как асинхронные приложения.
    API чат-бота выполнено также на языке Python с использованием веб-фреймворка Flask.
    Оба эти подпроекта используют диалоговую систему, в которой реализован непосредственный классификатор сообщений.
    Диалоговая система импортируется как библиотечный модуль через внутренний репозиторий программного обеспечения PyPI.

    \subsection{Основные понятия}
    \subsection{Обзор аналогов}