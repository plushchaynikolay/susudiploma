\section{АНАЛИЗ ПРЕДМЕТНОЙ ОБЛАСТИ}
    \subsection{Анализ исходных данных}
    Существующий проект выполняет задачи сбора сообщений с социальных сетей, мессенджеров и почты,
    их классификации и рассылки ответов пользователям.
    Проект оснащен системой непрерывной сборки и развертывания (CI/CD Pipeline), интегрированной в
    систему управления репозиториями \textit{GitLab}.
    Кодовая база проекта, как и сам проект, разделены на пять репозиториев с различными
    функциональными частями.
    
    Ниже представлен их список:
    \begin{enumerate}
        \item Приложения по работе с каналами
        \item API чат-бота для внешнего использования
        \item Диалоговая система чат-бота
        \item Система мониторинга и разметки сообщений
        \item Прокси-сервер и мониторинг нагрузки
    \end{enumerate}
    
    Все приложения собираются и разворачиваются внутри docker-контей-неров.
    Приложения по работе с каналами выполнены на языке \textit{Python} как асинхронные приложения
    с использованием библиотеки \textit{asyncio} и асинхронных библиотек.
    API чат-бота выполнено также на языке \textit{Python} с использованием веб-фреймворка \textit{Flask}.
    Оба эти подпроекта используют диалоговую систему, в которой реализован непосредственный
    классификатор сообщений.
    Диалоговая система импортируется как библиотечный модуль через внутренний репозиторий
    программного обеспечения \textit{PyPI}.
    При появлении обновления в репозитории диалоговой системы срабатывает триггер CI/CD пайплайна
    в приложениях каналов и API, и их конейнеры пересобираются с новой версией библиотеки.
    Система мониторинга реализована на языке \textit{Python} как \textit{Django} веб-приложение.
    Для мониторинга используется администраторская доска управления \textit{Django} приложения.
    В ней же реализован подсчет статистики и \textit{SQL-explorer} для запросов в базу данных.
    Прокси-сервер работает поверх всех сервисов, а именно: API-сервис, система мониторинга,
    системы оповещения и мониторинга ошибок \textit{Sentry}, система мониторинга нагрузки \textit{Prometheus}
    с выводом графики через \textit{Grafana}. Он перераспределяет http-запросы к поддоментам на конкретные
    приложения и сервисы.

    Функциональная схема проекта представлена на рисунке ниже.
    \begin{figure}[!h]
        \centering
        % \includegraphics[width=\linewidth]{}
        \caption{Функциональная схема исходного проекта}
        \label{fig:func-schema-before}
    \end{figure}
    
    Разберем подробнее основные функциональные элементы, это: приложение по работе с каналами, API-сервис и диалоговую систему.
    
    \subsubsection{приложение по работе с каналами}
    Приложение работает по трем каналам: Вконтакте, Телеграм и почта. Логику работы всех трех каналов можно обощить
    следующим образом: в цикле поочередно обрабатываются входящие  и исходящие письма, входящие письма дополнительно
    классифицируются, к ним вырабатывается и отправляется ответное сообщение, затем входящее или исходящее сообщение
    сохраняется в базу. Для сохранения в базу используется библиотека \textit{}, сохранение происходит через формирование
    сырого SQL-запроса.
    
    Для канала Вконтакте используется библиотека \textit{}, функциональная схема представлена ниже:
    \begin{figure}[!h]
        \centering
        % \includegraphics[width=\linewidth]{}
        \caption{Функциональная схема канала Вконтакте}
        \label{fig:func-schema-vk-before}
    \end{figure}

    Для канала Телеграм используется библиотека \textit{}, функциональная схема представлена ниже:
    \begin{figure}[!h]
        \centering
        % \includegraphics[width=\linewidth]{}
        \caption{Функциональная схема канала Телеграм}
        \label{fig:func-schema-vk-before}
    \end{figure}
    
    Для канала почты используется библиотека \textit{IMAPClient} и \textit{smtp}, функциональная схема представлена ниже:
    \begin{figure}[!h]
        \centering
        % \includegraphics[width=\linewidth]{}
        \caption{Функциональная схема канала почты}
        \label{fig:func-schema-vk-before}
    \end{figure}

    \subsubsection{API-сервис}


    \subsection{Анализ проблемных мест}
    После анализа исходного кода проекта мы можем выделить некоторые проблемные места, которые постараемся решить во время
    реинжениринга и рефакторинга. Ниже представлен полный список пробелем исходного проекта:
    \begin{enumerate}
        \item Сохранение в базу через сырой SQL-запрос небезопасно, подвергает систему к риску sql-иньекций кода, часто приводит
        к ошибкам во время исполнения, которые сложно отследить на этапе разработки.
        \item Каналы чат-бота имеют одинаковую логику, общность которой никак не описана. Код в целом имеет низкий уровень
        абстакции, частитчно дублируется, многие параметры работы каналов не вынесены в конфигурационные файлы, а жестко
        прописаны прямо в коде.
        \item 
        \item 
    \end{enumerate}

    % \subsection{План выполнения работ}
