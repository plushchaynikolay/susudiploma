\documentclass[14pt,oneside]{extarticle}
\usepackage[utf8]{inputenc}
\usepackage[english,russian]{babel}
\usepackage[backend=biber,style=numeric]{biblatex}
\addbibresource{src/bibliography.bib}
\usepackage{geometry}
\geometry{
    a4paper,
    total={170mm,257mm},
    left=30mm,
    top=20mm,
    right=15mm,
    bottom=20mm,
}

\usepackage{fancyhdr}
\pagestyle{fancy}
\renewcommand\headrulewidth{0pt} % no line between document and header
\fancyhead{} % clear header
\fancyfoot{} % clear footer
\fancyfoot[LE,RO]{\thepage}

\usepackage{listings}
\lstset{
    basicstyle=\footnotesize\ttfamily,
    frame=single,
    tabsize=4,
}
\newcommand{\inlinecode}{\lstinline[basicstyle=\normalsize\ttfamily]}

\usepackage{graphicx}
\usepackage{float}

\usepackage{indentfirst}
\parindent=1cm

\renewcommand{\labelenumii}{\theenumii}
\renewcommand{\theenumii}{\theenumi.\arabic{enumii}.}

\begin{document}
    \begin{titlepage}
    \begin{center}
        МИНИСТЕРСТВО И НАУКИ И ВЫСШЕГО ОБРАЗОВАНИЯ РОССИЙСКОЙ ФЕДЕРАЦИИ\\
        Федеральное государственное автономное образовательное\\
        учреждение высшего образования

        \textbf{
        <<Южно-Уральский Государственный университет\\
        (национальный исследовательский университет)>>\\
        Высшая школа электроники и компьютерных наук\\
        Кафедра системного программирования
        }
        \bigskip
        
        % \noindent
        % \newline
        % \begin{minipage}{0.5\textwidth}
        %     РАБОТА ПРОВЕРЕНА\\
        %     Рецензент,\\
        %     Программист ООО "Бонапарт"\\
        %     \underline{\hspace{2cm}} М.А. Тармин\\
        %     «\underline{\hspace{0.7cm}}» \underline{\hspace{2cm}} 2020 г.
        % \end{minipage}
        % \vspace{\fill}
        % \begin{minipage}{0.4\textwidth}
        %     ДОПУСТИТЬ К ЗАЩИТЕ\\
        %     Заведующий кафедрой,\\
        %     д.ф.-м.н., профессор\\
        %     \underline{\hspace{2cm}} Л.Б. Соколинский\\
        %     «\underline{\hspace{0.7cm}}» \underline{\hspace{2cm}} 2020 г.
        % \end{minipage}

        \vfill
        \large\textbf{
            ОТЧЕТ\\
            о выполнении научно-исследовательской работы
        }\\
        бакалавра направления 
        02.03.02 "Фундаментальная информатика и информацонные технологии"
        \bigskip
        
    \end{center}
    \vfill

    \hfill
    \begin{minipage}{0.4\textwidth}
        Автор работы,\\
        студент группы КЭ-220\\
        Н.В. Плющай\\
    \end{minipage}
    \bigskip
    \vfill

    \hfill
    \begin{minipage}{0.4\textwidth}
        Научный руководитель\\
        Доцент кафедры СП, к.т.н.\\
        Н.Ю. Долганина
        
        \noindent
        \newline
        
        Оценка: \underline{\hspace{2cm}}\\
        Дата: \underline{\hspace{2cm}}\\
        Подпись: \underline{\hspace{2cm}}
    \end{minipage}
    \bigskip
    \vfill

    % \hfill
    % \begin{minipage}{0.4\textwidth}
    %     Ученый секретарь\\
    %     (нормоконтролер)\\
    %     \underline{\hspace{2cm}} И.Д. Володченко\\
    %     «\underline{\hspace{0.7cm}}» \underline{\hspace{2cm}} 2020 г.
    % \end{minipage}
    % \bigskip
    % \vfill

    \begin{center}
        Челябинск-2020
    \end{center}
\end{titlepage}

    \begin{titlepage}
    
    \begin{center}
        МИНИСТЕРСТВО И НАУКИ И ВЫСШЕГО ОБРАЗОВАНИЯ РОССИЙСКОЙ ФЕДЕРАЦИИ\\
        Федеральное государственное автономное образовательное\\
        учреждение высшего образования

        \textbf{
        <<Южно-Уральский Государственный университет\\
        (национальный исследовательский университет)>>\\
        Высшая школа электроники и компьютерных наук\\
        Кафедра системного программирования
        }
        \bigskip
    \end{center}

    \hfill
    \begin{minipage}{0.4\textwidth}
        ДОПУСТИТЬ К ЗАЩИТЕ\\
        Заведующий кафедрой,\\
        д.ф.-м.н., профессор\\
        \underline{\hspace{2cm}} Л.Б. Соколинский\\
        «\underline{\hspace{0.7cm}}» \underline{\hspace{2cm}} 2020 г.
    \end{minipage}

    \begin{center}
        \textbf{ЗАДАНИЕ}\\
        \textbf{на выполнение выпускной квалификационной работы магистра}\\
        студенту группы КЭ-220 Плющай Николаю Владимировичу,\\
        02.04.02 «Фундаментальная информатика и информационные технологии»\\
        (магистерская программа «Технологии разработки высоконагруженных систем»)\\
    \end{center}

    \begin{enumerate}
        \item \textbf{Тема работы:} Реинжиниринг консультационного чат-бота для клиентов торговой компании
        \item \textbf{Срок сдачи студентом законченной работы:}
        % TODO Исходные данные к работе
        \item \textbf{Исходные данные к работе:}
        \begin{enumerate}
            \item Исходный код и документация существующего проекта.
        \end{enumerate}
        \item \textbf{Перечень подлежащих разработке вопросов}
        \begin{enumerate}
            \item Произвести анализ исходного проекта
            \item Выявить проблемные места
            \item Составить план проведения работ
            \item Изучить современные подходы и средства разработки
            % TODO уточнить
            \item Спроектировать систему чат-бота
            \item Реализовать, протестировать и развернуть новую систему
        \end{enumerate}
        % TODO Дата выдачи задания
        \item \textbf{Дата выдачи задания:}
    \end{enumerate}

\end{titlepage}

    \tableofcontents
    
    % введение
    \addcontentsline{toc}{section}{ВВЕДЕНИЕ}
\section*{ВВЕДЕНИЕ}

    % анализ предметной области
    \section{АНАЛИЗ ПРЕДМЕТНОЙ ОБЛАСТИ}
    \subsection{Анализ исходных данных}
    Существующий проект выполняет задачи сбора сообщений с социальных сетей, мессенджеров и почты, их классификации и рассылки ответов пользователям.
    Проект оснащен системой непрерывной сборки и развертывания (CI/CD Pipeline), интегрированной в систему управления репозиториями GitLab.
    Кодовая база проекта, как и сам проект, разделены на пять репозиториев с различными функциональными частями.
    
    Ниже представлен их список:
    \item Приложения по работе с каналами
    \item API чат-бота для внешнего использования
    \item Диалоговая система чат-бота
    \item Система мониторинга и разметки сообщений
    \item Прокси-сервер и мониторинг нагрузки

    Приложения по работе с каналами выполнены на языке Python как асинхронные приложения.
    API чат-бота выполнено также на языке Python с использованием веб-фреймворка Flask.
    Оба эти подпроекта используют диалоговую систему, в которой реализован непосредственный классификатор сообщений.
    Диалоговая система импортируется как библиотечный модуль через внутренний репозиторий программного обеспечения PyPI.

    \subsection{Основные понятия}
    \subsection{Обзор аналогов}
    % проектирование
    \section{ПРОЕКТИРОВАНИЕ}
    \subsection{Концептуальная модель обработки сообщений}
    Процесс обработки сообщения можно разделить на несколько этапов.
    \begin{enumerate}
        \item Получение сообщения через интерфейс (IMAP, LongPoll Вконтакте, Телеграм-api, json-файл из POST-запроса)
        \item Извлечение необходимых полей из инородной структуры данных и преобразование в обьект-сообщение внутреннего типа
        \item Классификация сообщения
        \item Формирование ответов на сообщение
        \item Отправка ответов обратно через внешний интерфейс пользователю
        \item Сохранение всех сообщений в базу
        \item Выполнение дополнительных действий (запрос обратной связи от пользователя, уведомления и прочее)
    \end{enumerate}

    Таким образом, за каждый из этапов обработки сообщения отвечать будет свой отдельный объект,
    а их взаимодействие будет происходить при переходе от одного этапа к другому.
    
    \subsection{Описание участников обработки сообщений}
    \subsubsection{ApiWorker}
    Объект, ответственный за первый этап - \textit{ApiWorker}. Ответственен за получение
    исходящих сообщений - от пользователя, и входящих сообщений - от оператора и бота.
    Его интерфейс очень прост - это одна функция, которая запускает процесс получения новых сообщений.
    Выглядит интерфейс так, как представлено на листинге ниже.
\begin{lstlisting}[language=Python]
class ApiWorker:
    def run(self):
        raise NotImplementedError
\end{lstlisting}

    \subsubsection{ApiWrapper}
    Объект, ответственный за взаимодействие с протоколами обмена сообщениями: IMAP, SMTP, VkLongPoll,
    Telegram-api и предоставляемого нами api.
    Он занимается превращением полученного сообщения в сообщение внутреннего формата проекта,
    обратным превращением перед отправкой сообщения пользователю и собственно отправкой.
    Его интерфейс выглядит следующим образом:
\begin{lstlisting}[language=Python]
class ApiWrapper:
    @classmethod
    def convert_from_api_message_type(cls,
                                      api_message: Any
                                      ) -> IMessageData:
        raise NotImplementedError

    def send_answers(self,
                     answers: List[Answer]
                     ) -> List[Answer]:
        ...

    def send_message(self, message: MessageData) -> MessageData:
        raise NotImplementedError
\end{lstlisting}
    Метод \lstinline{send_answers} отправляет все ответы из списка, дополняя их
    необходимой информацией от сервера текущего канала. 
    Метод \lstinline{send_message} отправляет одно отельное сообщения по текущему каналу,
    возвращает то же сообщение, но дополненное информацией от сервера: реальным временем
    отправки, идентификатором, который сервер канала обозначил это сообщение и так далее.

    \subsubsection{MessageHandler}
    Все действия между полученем сообщения в \lstinline{ApiWorker.run} и отправкой ответов
    в \lstinline{ApiWrapper.send_answers} и сохранением в базу будет скрывать в себе метод
    \lstinline{MessageHandler.handle}. Для входящих сообщений от пользователя обрабатывать
    сообщения будет IncomingHandler, для исходящих от оператора и бота - OutgoingHandler.
    Интерфейс класса MessageHandler представлен в листинге ниже.
\begin{lstlisting}[language=Python]
class MessageHandler:
    @staticmethod
    def notify(notifiers: Sequence[Type[Notifier]],
                message: IMessageData,
                scenarist_response: Optional[ResponseData] = None
                ) -> None:
        for notifier in notifiers:
            notifier.notify(message, scenarist_response)

    def handle(self,
                message: IMessageData,
                api_wrapper: ApiWrapper[IMessageData],
                notifiers: Sequence[Type[Notifier]] = ()
                ) -> None:
        raise NotImplementedError
\end{lstlisting}
    Метод \lstinline{MessageHandler.notify} здесь исполняет отправку всех уведомлений,
    которые были переданы ему. Под уведомлениями подразумеваются уведомления других служб
    о появлении новых сообщений, базы данных, телеграм-логгера и так далее.

    Обработчик входящих сообщений, IncomingHandler, также должен вырабатывать ответ
    к входящему сообщению. Эту задачу он делегирует другому классу - Scenarist.
    Дополнительно IncomingHandler может исполнять некоторые действия,
    в зависимости от выбранного сценария.
    Общий вид класса IncomingHandler представлен на листинге ниже.
\begin{lstlisting}[language=Python]
class IncomingHandler(MessageHandler):
    def __init__(self, scenarist: IScenarist[IMessageData]):
        self.scenarist: IScenarist[IMessageData] = scenarist

    def predict(self, message: IMessageData) -> ResponseData:
        ...

    @classmethod
    def send_answers(cls,
                        answers: List[Answer],
                        api_wrapper: ApiWrapper[IMessageData]
                        ) -> List[Answer]:
        ...

    @classmethod
    def do_actions(cls,
                    actions: Iterable[Action], 
                    api_wrapper: ApiWrapper[IMessageData]
                    ) -> None:
        for action in actions:
            action.do(api_wrapper)
\end{lstlisting}

    \subsubsection{Прочие компоненты}
    % TODO: Устройство и работа сценариста должны быть описаны более подробно:
    % извлечение слотов, 
    IScenarist - интерфейс сценариста, обладает только одним методом:\\
    \lstinline{IScenarist.run_scenario}, который при реализации, должен будет
    выработать нужную реакцию (ResponseData) на входящее сообщение с набором
    нужных ответов (Answer) и действий (Action).

    Задачу предсказания сценарист делегирует классификатору (Classifier),
    обладающему единственным методом \lstinline{Classifier.predict}, который
    возвращает объект Prediction с набором меток обнаруженных в тексте сообщения классов.
    По этим меткам и будет составлена ResponseData внутри сценариста.

    Работа с базой данных через объектно-реляционную модель (ОРМ) будет инкапсулирована
    в классе DBProvider.

    \subsection{Диаграммы}
    Теперь, когда мы выполнили описание всех участников обработки сообщений,
    можем для большей наглядности представить их на диаграмме классов. Сам процесс
    обработки одного сообщения изобразим на диаграмме процессов.

    Диаграмма классов представлена на рисунке ниже.
    \begin{figure}[!h]
        \centering
        % TODO
        % \includegraphics[width=\linewidth]{}
        \caption{Функциональная схема исходного проекта}
        \label{fig:func-schema-before}
    \end{figure}

    Диаграмма процесса обработки сообщения представлена на рисунке ниже.
    \begin{figure}[!h]
        \centering
        % TODO
        % \includegraphics[width=\linewidth]{}
        \caption{Функциональная схема исходного проекта}
        \label{fig:func-schema-before}
    \end{figure}

    \subsection{Вывод по главе 2}
    В этой главе мы сформулировали основную концепцию нашей будующей системы, описали основные
    её сущности и способы их взаимодействия. Мы в коде описали сигнатуры их публичных методов базовых
    классов, и изобралили основные классы и процессы на диаграммах.
    Теперь, мы можем приступать к их реализации в каждом конкретном канале.

    % реализация
    \section{РЕАЛИЗАЦИЯ}
    \subsection{Уведомления}
    Выполнение несрочных задач в проекте выполняется менеджером отложенных задач \textit(Celery):
    это широко известный кроссплатформенный проект с открытым исходным кодом на Python для
    асинхронного выполения задач через вызов удаленных процедур. В качестве брокера задач мы используем RabbitMQ.
    Для добавления задач в очередь в нашем проекте используются "уведомления" (\inlinecode{Notifier}).
    Перечислим все виды уведомлений, которые существуют на настоящий момент:
    \begin{itemize}
        \item \inlinecode{DataBaseMessageNotifier} -- сохранение сообщения в базу;
        \item \inlinecode{DataBaseResponseNotifier} -- сохранение ответов в базу;
        \item \inlinecode{DataBaseCarefulNotifier} -- сохранение сообщения в базу, если оно ещё не существует;
        \item \inlinecode{AppLogerNotifier} -- уведомление о некоторых событиях в мессенджер.
    \end{itemize}
    Пример реализации сохрания в базу представлен на листинге ниже.
\begin{lstlisting}[language=Python]
class DataBaseMessageNotifier(Notifier):
    @staticmethod
    def notify(message: MessageData,
               scenarist_response: Optional[ResponseData] = None
               ) -> None:
        celery_app.send_task('chat_bot.tasks.save_to_db',
                             args=(message.to_dict(),))
\end{lstlisting}

    Сама Celery-задача выглядит следующим образом:
\begin{lstlisting}[language=Python]
@celery_app.task(name='chat_bot.tasks.save_to_db')
def save_to_db(json_message: Mapping[str, Any]) -> None:
    message = MessageData.from_dict(json_message, [EmailMessageData])
    DBProvider.save(message)
\end{lstlisting}

    \subsection{Работа с базой}
    В работе с базой данных участвуют две сущности: объектно-реляционная модель Message
    и класс инкапсулирующий работу с базой DBProvider.
    Модель Message выглядит следующим образом:
\begin{lstlisting}[language=Python]
class Message:
    class Meta:
        db_table = 'messages'
        constraints = [
            models.UniqueConstraint(
                fields=['source', 'source_message_id'],
                name='messages_source_source_message_id_key')
        ]
    id = models.AutoField(primary_key=True)
    chat_id = models.CharField(max_length=50, null=True)
    source = models.CharField(max_length=50, null=False)
    datetime = models.DateTimeField(null=False)
    text = models.TextField(null=True)
    sender = models.CharField(max_length=10, null=False)
    is_answered = models.BooleanField(null=True)
    prediction = JSONField(null=True)
    correctly_answered = models.BooleanField(null=True)
    categories = ArrayField(models.IntegerField(), null=True)
    source_message_id = models.CharField(max_length=50, null=True)
\end{lstlisting}

    DBProvider -- это довольно большой статический класс, реализующий как получение исории сообщений,
    так и сохранение новых сообщений в базу.
    Структура класса приведена на листинге ниже.
\begin{lstlisting}[language=Python]
class DBProvider:
    def get_history(cls, chat_id: str, 
                    columns: Iterable[str]) -> Optional[History]: ...
    def _load_history(chat_id: str, 
                      columns: Iterable[str]) -> History: ...
    def get_latest_datetime(**kwargs: str
                            ) -> Optional[timezone.datetime]: ...
    def save(cls, message: MessageData) -> None: ...
    def get_or_create(cls, message: MessageData) -> bool: ...
    def update_or_create(cls, message: MessageData) -> bool: ...
    def _save(message: MessageData, 
              need_check: bool = False, 
              update: bool = False) -> bool: ...
\end{lstlisting}

    Рассмотрим подробнее методы этого класса:
    \begin{itemize}
        \item \inlinecode{get_history} -- публичный метод для получения истории диалога, осуществляет
        безопасную выгрузку истории чата из базы данных, обрабатывая возникающие исключения;
        \item \inlinecode{_load_history} -- приватный метод, реализующий непосредственную загрузку истории из базы,
        устанавливает временной диапазон истории и формирует результат в надлежащий вид;
        \item \inlinecode{get_latest_datetime} -- метод получения информации о времени последнего сообщения
        в канале или чате, от пользователя, оператора или бота;
        \item \inlinecode{save} -- сохранение нового сообщения в базе, реализует SQL-функцию \inlinecode{INSERT};
        \item \inlinecode{get_or_create} -- создание записи о сообщении в базе, если в этом канале сообщения с таким
        идентификатором ещё не существует, реализует SQL-функцию \inlinecode{IF NOT EXISTS INSERT};
        \item \inlinecode{update_or_create} -- обновление записи в базе о сообщении определенного канала с таким
        идентификатором, либо создание новой записи, реализует SQL-функцию \inlinecode{IF EXISTS UPDATE ELSE INSERT};
        \item \inlinecode{_save} -- непосредственная реализация три перечисленные выше функции, выбирая нужный
        метод библиотеки \textit{Django} подставляет нужные фильтры поиска и значения для создания
        или обновления записи.
    \end{itemize}

    \subsection{Датаклассы}
    С версией \textit{Python 3.8} в синтаксис языка были введены так называемые
    датаклассы (\textit{dataclasses}).
    Это усовершенствованный синтаксис для описания собственных типов и структур данных.
    В проекте существуют несколько представителей датаклассов. Опишем отдельно каждого из них.
    \begin{enumerate}
        \item \inlinecode{MessageData} -- датакласс для описания сообщений входящих и исходящих,
        содержит всю необходимую для работы обработчика, сценариста и других классов информацию:
        идентификатор чата, название канала-источника, время получения, отправителя,
        список классифицированных категорий, а также некоторые флаги и идентификаторы.
\begin{lstlisting}[language=Python]
@dataclass
class MessageData(SerializableMixin):
    chat_id: str = ''
    source: str = ''
    timestamp: Optional[datetime] = None
    text: str = ''
    sender: Optional[str] = None
    is_answered: Optional[bool] = None
    categories: List[int] = field(default_factory=list)
    prediction: Optional[Dict[str, float]] = None
    source_message_id: Optional[str] = None
    source_chat_id: Optional[str] = None
\end{lstlisting}

        \item \inlinecode{Answer} -- контейнер с ответом бота на входящее сообщение.
\begin{lstlisting}[language=Python]
@dataclass
class Answer(SerializableMixin):
    category: str
    message: MessageData
\end{lstlisting}

        \item \inlinecode{Prediction} -- контейнер с результатами классификации текста сообщения,
        которые возвращает классификатор.
\begin{lstlisting}[language=Python]
@dataclass
class Prediction(SerializableMixin):
    categories: List[str]
    probabilities: Optional[Dict[str, float]]
\end{lstlisting}

        \item \inlinecode{ResponseData} -- датакласс со всем, что нужно боту, чтобы корректно
        среагировать на входящее сообщение пользователя.
\begin{lstlisting}[language=Python]
@dataclass
class ResponseData(SerializableMixin):
    prediction: Prediction = field(default=Prediction([], None))
    answers: List[Answer] = field(default_factory=list)
    after_answer_actions: List[Action] = field(
        default_factory=list)
\end{lstlisting}
    \end{enumerate}

    \subsection{Канал Телеграма}
    Так как за обработку отложенных действий теперь отвечает менеджер задач \textit{Celery},
    исчезла потребность в асинхронной обработке сообщений, мы отказались от асинхронности.
    Это позволяет использовать синхронные библиотеки, выбор которых гораздо больше,
    а логика синхронных приложений гораздо проще для понимания.
    В новой реализации Телеграм-канала была использована другая библиотека -- \textit{pyrogram}.
    Структура \inlinecode{TelegramWorker} осталась примерно той же: два фильтра событий и две функции, описывающие
    обработку входящих и исходящих сообщений подаются в цикл обработки событий (\textit{event loop})
    телеграм клиента.
    Обработка сообщений заключается в их преобразовании во внутренний тип \inlinecode{MessageData}
    и передаче обработчику в \inlinecode{IncomingHandler} или в \inlinecode{OutgoingHandler}.
    Класс \inlinecode{TelegramApiWrapper} наследуется от базового класса \inlinecode{ApiWrapper},
    реализует его абстрактные методы \inlinecode{send_message} и \inlinecode{convert_from_api_message_type}.

    \subsection{Канал Вконтакте}
    Для реализации канала Вконтакте используется та же библиотека \textit{vk\_api}. Для получения сообщений используется
    технология LongPoll, в которой, после установления соединения и начала сессии, клиент получает
    от сервера Вконтакте уведомления о всех новых событиях в беседе. У каждого нового события проверяется
    является ли оно сообщением, а также поле \inlinecode{to_me}, из которого определяется каким образом
    должно обрабатываться сообщение, как входящее или исходящее.
    Обработка сообщений, также как и в случае с телеграмом, заключается в их преобразовании во внутренний тип
    и передаче одному из обработчиков.

    \subsection{Канал почты}
    Ввиду отсутствия у библиотек по работе с почтой инструментов, похожих на LongPoll, этот инструмент был
    создан самостоятельно. Для этого мы поочередно просматриваем папки входящей и исходящей почты на наличие
    непрочитанных и необработанных ботом сообщений. Также был объявлен специальный датакласс \inlinecode{EmailMessage}
    канала почты, со специальными полями, и класс \inlinecode{EmailMessageBuilder} для создания объекта датакласса.
    Дополнительно были созданы классы обертки над библиотеками \textit{imap} и \textit{smtp}, в которых
    реализован алгоритм установления и проверки соединения и переподключения в случае его разрыва.
    На почте присутсвует сложная логика, связанная пометкой обработанных ботом сообщений флажком
    и перекладыванием отвеченных сообщений в отдельную папку. Дополнительно ситуация осложняется тем, что
    одновременно в одном почтовом ящике могут работать кроме бота ещё два оператора и сторонний парсер почты,
    из-за чего часто возникают конфликтные ситуации. Ситуации могут выражаться, например в прочтении и снятии флага
    \inlinecode{unseen} с писем, предназначенных для бота, и так далее. Поэтому обработка писем в этом канале перегружена
    различными дополнительными проверками.

    \subsection{Сценарист и сценарии}
    Диалоговая система в целом не перенесла существенных изменений, так как целью рефакторинга не ставилось
    улучшение точности или полноты распознавания. Изменения претерпел только модуль для сборки модели.
    Он был перенесен в рабочий проект целиком, практически без изменений, в корневую директорию нового проекта.
    В перую очередь, это было нужно, чтобы решить проблему импортов библиотек при распаковке модели.
    Дело в том, что для сериализации модели в проекте используется библиотека \textit{joblib}, которая при сериализации
    классов, использованных в составе модели, сохраняет путь импортов, а при десериализации использует эти же пути
    для восстановления модели в оперативной памяти.
    Проблема заключается в том, что пути импортов сохраняются отностильно точки запуска скрипта сборки, или,
    в нашем случае, \textit{jupyter} тетрадки с кодом сборки. Соответственно, если точка входа в приложение
    будет находится в другом каталоге, импорты провалятся и приложение завершится с ошибкой.
    Размещение скриптов для сборки модели и запуска приложения в одном каталоге -- самое легкое и очевидное
    решение, в дальнейшем позволяет не отвлекаться на проблему загрузки модели во время работы над ней.

    \subsection{Обработчик}
    Для обработки сообщений был реализован базовый класс \inlinecode{MessageHandler}, в котром объявлен
    абстрактный метод \inlinecode{handle()} и реализован общий для входящих и исходящих сообщений
    метод \inlinecode{notify}, который рассылает уведомления всем заявленным уведомителям.
    Метод \inlinecode{handle()} содержит реализуется в виде перечня последовательных действий,
    которые должны быть проделаны над сообщением.

    В \inlinecode{IncomingHandler} это вызов сценариста для формирования ответа, отправка ответных сообщений,
    отправка уведомлений, выполнение дополнительных действий. Листинг метода представлен ниже.
\begin{lstlisting}
class IncomingHandler(MessageHandler):
    def handle(self, message: IMessageData,
               api_wrapper: ApiWrapper[IMessageData],
               notifiers: Sequence[Type[Notifier]] = ()) -> None:
        response = self.predict(message)
        response.answers = self.send_answers(response.answers, api_wrapper)
        self.notify(notifiers, message_with_prediction, response)
        self.do_actions(response.after_answer_actions, api_wrapper)
\end{lstlisting}

    В \inlinecode{OutgoingHandler} никаких действий кроме отправки уведомлений не требуется.
    Реализация этого класса представлена ниже.
\begin{lstlisting}
class OutgoingHandler(MessageHandler):
    def handle(self, message: IMessageData,
               notifiers: Sequence[Type[Notifier]] = ()) -> None:
        self.notify(notifiers, message)
\end{lstlisting}

    \subsection*{Вывод по главе 3}
    В этой главе мы описали все реализованные нами классы и методы, объяснили принципы
    их работы и перечислили некоторые связанные с ними нюансы. Для наглядности мы привели
    куски кода.
    В ходе самой разработки были решены основные проблемы, перечисленные в первой главе данной работы.
    Проект переписан с единой архитектурой для всех каналов чат-бота что улучшает понимание и поддержку
    проекта, для работы с базой теперь используется объектно-реляционная модель, что также сильно
    упрощает поддержку и написание нового функционала,
    проект полностью снабжен аннотациями типов, что позволяет проводить статический анализ.

    % тестирование
    \section{ТЕСТИРОВАНИЕ}
    \subsection{Юнит-тесты}


    \subsection{Нагрузочное тестирование}
    Нагрузочное тестирование проводилось на тестовом сервере на полностью
    развернутом веб-приложении с веб-сервером \textit{Nginx} при помощи
    инструмента Яндекс.Танк. Целью нагрузочного тестирования было установление
    порога стабильной работы сервиса, для этого на него подавалась нагрузка
    с константным количеством запросов в секунду в течение 2 минут с итеративным
    увеличением количества запросов.

    Так, был составлен следующий конфигурационный файл для Яндекс.Танка:
    \begin{lstlisting}
# load.yaml
phantom:
    address: alisa.napoleonit.ru:443
    ammofile: ./ammo.txt
    timeout: 31s
    ssl: true
    load_profile:
        load_type: rps 
        schedule: const(20, 1m)
console:
    enabled: true 
    \end{lstlisting}

    Так, было установлено, что сервис стабильно выдерживает нагрузку до 18
    запросов в секунду в течение 2 минут.
    % TODO какой-нибудь график

    % сравнительный анализ сложности проектов
    \section{АНАЛИЗ ПРОЕКТОВ}
    \subsection{Метрики}
    % TODO Метрики
    Целью реинжениринга проекта было улучшение его ясности, надежности и поддерживаемости,
    модифицируемости, для этого нужно максимально уменьшить сложность и сцепленность кода,
    увеличить покрытие кода тестами.
    Для оценки успешности выполненных работ, применим метрики размера,
    стилистики и сложности программы.
    \cite{metrics.cmcons}\cite{clearcode.habr}

    Для оценки изменений в объеме кодовой базы подсчитаем количества строк
    (LOC-метрика, \textit{lines of code}) кода в проекте.
    Для оценки стилистики и понятности подсчитаем количество модулей и средний
    объем модуля -- это даст нам понять, насколько изменилась читаемость кода,
    ведь чем больше текст программы, тем сложнее удержать во внимании суть при 
    её изучении. Также подсчитаем количество пустых строк, чтобы оценить
    "разряженность" текста программы.
    При этом мы не станем подсчитывать степень документированности кода,
    так как в текущей команде разработчиков не поощряется комментирование.
    Такое решение было принято из тех принципов,
    что код должен быть понятен без комментариев,
    что отсутствие комментариев вынуждает программиста придумывать лучшие имена
    для переменных и функций,
    и что комментарии необходимо регулярно актуализировать, а делать это
    гораздо сложно, так как комментарии, в отличие от кода,  нельзя
    протестировать.
    Можно добавить, что в рекомендациях PEP отмечается, что отсутствующий
    комментарий лучше неправильного.

    Цикломатическая сложность программы - количество независимых маршрутов через
    неё. Вычисляется с помощью ориентированного графа, узлами которого являются
    блоки программы, соединенные рёбрами, если управление может переходить с
    одного блока на другой. Определяется как:
    \begin{center}
        M = E - N + 2P,
    \end{center}
    где:
    \begin{description}
        \item [M] - цикломатическая сложность,
        \item [E] - количество рёбер в графе,
        \item [N] - количество узлов в графе,
        \item [P] - количество компонент связности.
    \end{description}
    Цикломатическая сложность любой структурированной программы с только одной
    точкой входа и одной точкой выхода эквивалентна числу точек ветвления,
    то есть, операторов if или условных циклов, плюс один.
    \cite{complexity.McCabe}

    \subsection{Сравнение}
    % TODO Сравнение
    Для подсчета количества строк, пустых строк, модулей классов и интерфейсов
    воспользуемся unix утилитами \textit{ls, wc, grep}.

    Подсчет цикломатической сложности будем осуществлять при помощи утилиты для
    анализа Python приложений \textit{radon} \cite{complexity.radon}.
    Для сравнения в таблице возьмем среднюю сложность по всем модулям, классам
    и функциям проекта.
    
    \begin{table}[H]
        \caption{Сравнение метрик программного обеспечения проектов}
        \begin{center}
            \begin{tabular}{l|r|r}
                \textbf{Метрика} & \textbf{Исходный} & \textbf{Конечный} \\
                \hline
                Количество строк кода               & 7436 & 5886 \\
                Количество пустых строк             & 1077 & 871 \\
                Разряженность, \%                   & 14.48 & 14.8 \\
                Количество модулей                  & 67 & 87 \\
                Среднее число строк на модуль       & 110.99 & 67.66 \\
                Количество классов и интерфейсов    & 100 & 182 \\
                Цикломатическая сложность           & 2.4903 & 2.1176 \\
                Анализ функциональных точек         & & \\
                Покрытие кода тестированием         & & \\
                Покрытие требований                 & & \\
                Метрики программного пакета от      & & \\
                Роберта Сесиль Мартина              & & \\
                Связность                           & & \\
            \end{tabular}
        \end{center}
    \end{table}

    \subsection*{Вывод по главе 6}
    % TODO Вывод по главе 6

    % заключение
    \addcontentsline{toc}{section}{ЗАКЛЮЧЕНИЕ}
\section*{ЗАКЛЮЧЕНИЕ}


    % литература
    % FIXME стиль списка литературы
    \addcontentsline{toc}{section}{Библиографический список}
    \printbibliography[title={Библиографический список}]

\end{document}